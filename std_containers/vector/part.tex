\subsection{vector}

\lstinline|vector| представляет собой упорядоченный список элементов одинакового типа.
Создавать его можно разными способами:

\begin{lstlisting}
#include <vector>
using namespace std;

int main()
{
    vector<int> a;
    vector<pair<int, int>> b(10);
    vector<char> c(15, 'a');
    vector<int> d = {1, 2, 3};
}
\end{lstlisting}
Сверху создались три \lstinline|vector|. Первый --- пустой вектор, способный содержать в себе целые числа. Второй --- вектор из десяти пар целых чисел (все числа изначально --- нули). Третий --- вектор, содержащий пятнадцать букв \lstinline|'a'|. Четвёртый --- состоит из целых чисел \lstinline|1, 2, 3|.


Определим вектор и число:
\begin{lstlisting}
vector<int> numbers = {1, 2, 3};
int number = 4;
\end{lstlisting}
Векторы поддерживает следующие основные операции (больше можно найти на \href{https://ru.cppreference.com/w/cpp/container/vector}{cppreference}):

\begin{table}[h]
    \begin{tabular}{|l|c|c|}
    \hline
    \multicolumn{1}{|c|}{Операция} & Сложность & Пример \\ \hline
    Добавление элемента в конец списка &  $O(1)$ & \lstinline|numbers.push_back(number)| \\ \hline
    Удаление последнего элемента & $O(1)$ & \lstinline|numbers.pop_back()| \\ \hline
    Просмотр произвольного элемента & $O(1)$ & \lstinline|numbers[i]| \\ 
    \hline
    \end{tabular}
\end{table}

