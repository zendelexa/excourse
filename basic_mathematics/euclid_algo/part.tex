\subsection{Алгоритм Евклида}

% \emph{Наибольшим общим делителем} чисел $x$ и $y$ называется максимальное целое число, нацело делящее и $x$, и $y$.

Задача о поиске \emph{наибольшего общего делителя (НОД, или GCD)} и \emph{наименьшего общего кратного (НОК, или LCM)} нескольких чисел нередко возникает при упрощении математических моделей предметов с циклическими свойствами. 
% Как видно из названия, НОД нескольких чисел определяется как наибольшее целое число, нацело делящее каждое из них, а НОК ---

Для доказательства работоспособности достаточно эффективного вычисления НОД двух чисел приведём следующие рассуждения. 
% Заметим следующий факт: если число $d$ --- общий делитель чисел $x$ и $y$, то он также будет делителем значений $x + y$ и $x - y$. Также убедимся в том, что набор общих делителей двух чисел $x$ и $y$ совпадает с набором общих делителей. 
Пусть есть целые неотрицательные числа $x$ и $y$. Не умаляя общности, обозначим их так, что $x \ge y$. Если число $d$ --- их некоторый общий делитель, то он также будет делителем $(x - y)$. Также и наоборот, если $d$ --- общий делитель $(x - y)$ и $y$, то он также и общий делитель $x$ и $y$. Получается, что при замене $x$ на $(x - y)$ общие делители не пропадают и не появляются. В частности \[\gcd(x, y) = gcd(x - y, y).\]
Это рекуррентное выражение можно преобразовать, если расписать $x$ как $q \cdot y + r$, где $q$ и $r$ --- целая часть и остаток от деления $x$ на $y$. Кратно применив вышеописанную формулу, получаем:
\[\gcd(x, y) = \gcd(q \cdot y + r, y) = \gcd(q \cdot y + r - y, y) = \gcd((q-1) \cdot y + r, y) =\]\[= \gcd((q-1) \cdot y + r - y, y) = \gcd((q-2) \cdot y + r, y) = \dots =\]\[= \gcd((q-q) \cdot y + r, y) = \gcd(r, y).\]
Выходит, что в случае двух аргументов больший можно заменить остатком от деления на меньший. После нескольких таких замен неизбежно придём к тривиальному случаю $\gcd(0, w) = w$. Запишем этот алгоритм, называемый \emph{алгоритмом Евклида}.
\begin{lstlisting}
ll gcd(ll x, ll y)
{
    if (x < y) return gcd(y, x);

    if (y == 0) return x;

    return gcd(y, x % y);
}
\end{lstlisting}

Если числа $x$ и $y$ в своих разложениях имеют общий делитель $d$, то он обязательно присутствует в разложении $\gcd(x, y)$. Также, если хотя бы в одном разложении $x$ или $y$ имеется делитель $d$, то он есть и в разложении $\text{lcm}(x, y)$. Нетрудно заметить, что в произведении $x \cdot y$ общие делители будут дублироваться, а уникальные появятся единожды. Из этого следует равенство:
\[x \cdot y = \gcd(x, y) \cdot \text{lcm}(x, y).\]
Тогда формула для НОК принимает вид \(\text{lcm}(x, y) = \frac{x \cdot y}{\gcd(x, y)}\).